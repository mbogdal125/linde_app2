\documentclass[8pt]{article}
\usepackage[top=1.5cm, bottom=1.5cm, left=1.5cm, right=1.5cm,a4paper]{geometry}
\usepackage{multicol}
\usepackage{tabularx}
\setlength\parindent{0pt}
\begin{document}

\begin{center}
{\large{\bf ARKUSZ INWENTARYZACYJNY NR}}
\end{center}

\begin{multicols}{2}

Adres Dostawy:

a: %%ClientName%%

b: %%ClientStreet%%

c: %%ClientCity%%

\vspace{10mm}

1. \dotfill \\


2. \dotfill \\


3. \dotfill \\

\vspace{-5mm}
{\tiny(Nazwa i adres jednostki/stempel)} \\

Sklad komisji: \\
{\tiny (Imie i nazwisko/stanowisko sluzbowe)} \\

1. \dotfill \\


2. \dotfill \\


3. \dotfill \\

\begin{tabularx}{\textwidth}{l l}
Spis rozpoczeto dnia: \dotfill & godz: \dotfill \\
\end{tabularx}

\columnbreak

Metoda przeprowadzenia: %%Method%%

NUMER KLIENTA: %%ClientNumber%%

\vspace{30mm}

Numer ostatniego dowodu dostawy lub faktury \dotfill

\vspace{5mm}

{\bf Osoby uprawnione do reprezentowania firmy}\\
{\tiny(Imie i nazwisko/stanowisko sluzbowe)} \\

1. \dotfill \\

2. \dotfill \\

\end{multicols}

\begin{center}

\begin{tabularx}{\textwidth}{| X | c |}

\hline

ta linia zostanie powtorzona pare razy dla typu butli: %%for:BottleTypes%% & \\ \hline

\end{tabularx}

\end{center}

\begin{multicols}{2}

Podpisy komisji:

1. \dotfill \\


2. \dotfill \\


3. \dotfill \\

\columnbreak

Podpis osoby uprawnionej do reprezentowania firmy:

1. \dotfill \\


2. \dotfill \\

Podpis osoby odpowiedzialnej materialnie:

1. \dotfill \\

\end{multicols}

\end{document}
